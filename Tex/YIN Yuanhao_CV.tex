\documentclass[french]{../crudecv/crudecv}
\usepackage[utf8]{inputenc}
\usepackage{lipsum}
\usepackage{blindtext}
\usepackage{multicol}
\usepackage{hyperref}
\usepackage{xfrac}
\usepackage{glossaries}
% 自己的包
\usepackage{enumitem}

\author{YIN Yuanhao}
\applicant{28/05/1998}{9 rue Sophie Germain 75014 Paris}{yuanyang11510@gmail.com}{+33 7 88 03 23 92}{https://github.com/yuanyang11510}

% \newacronym{ntnu}{NTNU}{Norwegian University of Science and Technology}
\newacronym{inalco}{INALCO}{Institut National des Langues et Civilisations Orientales}
\newacronym{amu}{AMU}{Université d'Aix-Marseille}
\newacronym{swupl}{SWUPL}{Southwest University of Political Science and Law}
\newacronym{sisu}{SISU}{Sichuan International Studies University}
\newacronym{crlao}{CRLAO}{Centre de recherches linguistiques sur l'Asie orientale}
\newacronym{efl}{EFL}{Laboratory of Excellence “Empirical Foundations of Linguistics”}
\newacronym{htl}{HTL}{Histoire des théories linguistiques}
\newacronym{ehess}{EHESS}{École des hautes études en sciences sociales}
\newacronym{cnrs}{CNRS}{Centre national de la recherche scientifique}

\begin{document}
\maketitle

\section*{Profil}
Diplômé en 2023 d’un Master en sciences du langage à l’\gls*{inalco}. Doctorant en sciences du langage à l’\gls*{inalco}, sous la direction de M. Guillaume JACQUES et M. Thomas PELLARD, avec une thèse en cours intitulée : Phylogénie des langues sinitiques. Je m'intéresse à la programmation (R,etc.) et ses applications au traitement automatique des données linguistiques.

\section*{Parcours professionnel}

\begin{experiences}
  \experience{Assistant de langue chinoise\\(Lycée Marcel Pagnol \& Collège les Bartavelles, Marseille)}{
    \begin{itemize}[topsep = 2pt,itemsep = 0pt,parsep = 0pt]
      \item J'aidais le professeur à enseigner le chinois à un groupe d’élèves avancés pendant que le professeur s'occupait d'un autre groupe dans une autre salle.
      % \item J'apportais au professeur des connaissances culturelles sur la société chinoise lors des cours.
      % \item Je recherchais des documents pédagogiques demandés par le professeur.
      % \item J'organisais, en collaboration avec le professeur, les activités culturelles pour le Nouvel An chinois.
    \end{itemize}
  }{
    12/2020
  }{
    04/2021
  }
  \experience{Annotateur et transcripteur linguistique du wu de Suzhou à temps partiel\\(Association culturelle de Suzhou, Université de Suzhou)}{
    \begin{itemize}[topsep = 2pt,itemsep = 0pt,parsep = 0pt]
      \item Segmentation d'enregistrement audio et transcription en caractères chinois en dialecte wu de Suzhou.
    \end{itemize}
  }{
    06/2020
  }{
    10/2020
  }
  \experience{Stagiaire auprès du procureur\\(Parquet populaire de Kecheng, Quzhou, Zhejiang)}{
    \begin{itemize}[topsep = 2pt,itemsep = 0pt,parsep = 0pt]
      \item J'assistais le procureur dans le traitement des dossiers relevant du pénal.
      % \item J'ai acquis des connaissances en matière d’enquête et de surveillance.
    \end{itemize}
  }{
    07/2018
  }{
    09/2018
  }
\end{experiences}
\vspace{-5pt}
\section*{Formation}
\begin{experiences}
  \experience{Master en sciences du langage}{
    \begin{itemize}[topsep = 2pt,itemsep = 0pt,parsep = 0pt]
      \item Parcours: Linguistique et diversité des langues
      \item Etablissement: \gls*{inalco}
    \end{itemize}
  }{
    09/2021
  }{
    06/2023
  }
  \experience{Séjour en France (étudiant d'échange)}{
    \begin{itemize}[topsep = 2pt,itemsep = 0pt,parsep = 0pt]
      \item Etablissement: \gls*{amu}
    \end{itemize}
  }{
    09/2019
  }{
    06/2020
  }
  \experience{Licence en droit et en français (double diplôme)}{
    \begin{itemize}[topsep = 2pt,itemsep = 0pt,parsep = 0pt]
      \item Etablissement: \gls*{swupl} \& \gls*{sisu}
    \end{itemize}
  }{
    09/2016
  }{
    06/2020
  }
\end{experiences}

\section*{Engagements et bénévolat}
\begin{experiences}
   \experience{Représentant des doctorant(e)s du \gls*{crlao}}{
    \begin{itemize}
      % [topsep = 2pt,itemsep = 0pt,parsep = 0pt]
      % \item Soutien aux responsables du laboratoire pour l’organisation des réunions avec les doctorant(e)s.
      \item Information et accompagnement des doctorant(e)s sur les démarches administratives.
      \item Transmission des besoins et des retours des doctorant(e)s auprès du laboratoire.
    \end{itemize}
  }{
    09/2024
  }{
    présent
  }
  \experience{Bénévolat pour la promotion de Campus France}{
    \begin{itemize}
      % [topsep = 2pt,itemsep = 0pt,parsep = 0pt]
      \item Distribution de flyers sur le campus du \gls*{swupl} et diffusion d’informations sur les sites web pour la promotion de Campus France.
    \end{itemize}
  }{
    09/2018
  }{
    10/2018
  }
\end{experiences}

\section*{Participation à l'organisation de colloques}
\begin{itemize}
  % [topsep = 2pt,itemsep = 0pt,parsep = 0pt]
  \item 04 -- 06/09/2023 \hspace{5pt} 26th Himalayan Languages Symposium (HLS 26)
  \begin{itemize}
    \item soutenu par \gls*{inalco}, \gls*{efl}, \gls*{htl} et \gls*{crlao}
  \end{itemize}
  \item 08 -- 10/07/2025 \hspace{5pt} 38e Journées de linguistique d'Asie orientale (JLAO 38)
  \begin{itemize}
    \item soutenu par \gls*{crlao}, \gls*{inalco}, \gls*{ehess}, \gls*{cnrs} et Campus Condorcet
  \end{itemize}
\end{itemize}

\section*{Langues}
\begin{itemize}
  % [topsep = 2pt,itemsep = 2pt,parsep = 2pt]
\item Chinois (wu): Langue maternelle 1 (acquis naturellement)
\item Chinois (mandarin): Langue maternelle 2 (appris en milieu scolaire)
\item Français: C1
\item Anglais: B2
\end{itemize}

\section*{Langages de programmation}
\begin{skills}
  \skill{R}{4}
  \skill{\LaTeX}{4}
  \skill{Phyton}{2}
\end{skills}

% \section*{Projets}
% \textbf{Yuanhao\_Phd\_Thesis
% /Proto\_Min:}
% % \href{https://github.com/yuanyang11510/Yuanhao_Phd_Thesis/tree/main/Proto_Min}

% \vspace{5pt}
% \noindent Un dépôt personnel (pas encore public) dans le cadre de ma thèse visant à traiter automatiquement les données du proto-min vers divers dialectes du min de nos jours, comprenant 
% \begin{itemize}[topsep = 2pt,itemsep = 2pt,parsep = 2pt]
%   \item l’établissement des proto-formes et des règles de correspondance.
%   \item l’application des changements phonétiques conditionnés ainsi que leur ordre.
%   \item la déduction des proto-formes vers des réflets théoriques.
%   \item la comparaison entre ces réflets théoriques et les formes attestées.
% \end{itemize}

\end{document}
